\documentclass[a4paper]{article}

\usepackage[english]{babel}
\usepackage[utf8]{inputenc}
\usepackage{amsmath}
\usepackage{graphicx}
\usepackage[numbered]{bookmark}
\usepackage[colorinlistoftodos]{todonotes}
\usepackage{algorithm}
\usepackage{algpseudocode}
\usepackage{pifont}
\usepackage{tikz}
\usepackage{bm}


\title{Coherent Detection Systems for Data Centers}

\author{JKP}

\date{\today}

\begin{document}
\maketitle

\section{Transmitter}
\subsection{Intensity Noise}
Modeled as an AWG noise added to the optical power at the transmitter.

The value of RIN is defined as the ratio between the noise power divide by the noise bandwidth and the signal power \cite{agilent-RIN-measurement}: 
\begin{equation}
RIN = \frac{P_{noise}}{B_{noise}P_{signal}}
\end{equation}

Thus the \textbf{one-sided RIN PSD} and \textbf{RIN variance} at a certain instant are given by
\begin{align}
& S_{RIN}(t) = RIN\cdot P(t)^2 \\
& \sigma^2_{RIN}(t) = S_{RIN}(t)\frac{f_{s, sim}}{2}
\end{align}
where $f_{s, sim}$ is the sampling frequency to simulate continuous time. Obviously, the variance as defined here only make sense in simulations. Since the intensity noise is assumed to be white, it'd have infinite variance.

Output optical power $P(t)$ is given by
\begin{equation}
P(t) = P_s(t) + w_{RIN}(t)
\end{equation}
where $P_s(t)$ is the signal-only optical power (after modulator frequency response and extinction ratio), and $w_{RIN}(t)\sim\mathcal{N}(0, \sigma^2_{RIN}(t))$.

\subsection{Phase Noise}
Phase noise is modeled as a Wiener process (Gaussian-distributed independent increments).

\begin{equation}
\phi(t+\tau) - \phi(t) \sim\mathcal{N}(0, 2\pi\Delta\nu\tau),
\end{equation} 
where $\Delta\nu$ is the laser linewidth. 

The coherence time is defined as a delay difference yielding an rms value for the phase change of $\sqrt{2}$ rads.

\subsection{Modulator Bandwidth Limitations}

\subsubsection{Mach-Zenhder modulator}
\cite{Barros2009, Ho2005}

Limited by loss and velocity mismatch

\begin{equation}
H_{mod}(f) = \frac{1-e^{-\alpha(f)L+j2\pi fd_{12}L}}{\alpha(f)L-j2\pi fd_{12}L}
\end{equation}
where $\alpha(f)$ is the frequency-dependent loss, $d_{12}$ is the velocity mismatch between the optical and electrical waveguides, and $L$ is the interaction length.

\begin{equation}
d_{12} = \frac{n_m-n_r}{c}
\end{equation}
where $n_r \approx 2.15$ is the refractive index of the coplanar waveguide for TM input light. If $n_m$ is only $95\%$ of $n_r$ a significant reduction in bandwidth may occur due to velocity mismatch \cite{Ho2005}.

\subsubsection{Silicon photonics modulator}
Currently modeled as a second-order system with unit damping. This is based on the assumption that parasitics is the limiting factor in the bandwidth of these devices.

Second-order system with unit damping:
\begin{equation}
H_{mod}(f) =  \frac{1}{1 + 2jf/f_c - (f/f_c)^2}
\end{equation}

The modulator bandwidth is related to $f_c$ by
\begin{equation}
f_{3dB} = \sqrt{\sqrt{2}-1}f_c = 0.64359f_c
\end{equation}

Group delay:
\begin{equation}
\Delta\tau_g = \frac{2}{2\pi f_c}
\end{equation}

\section{Fiber Propagation}
\subsection{Chromatic Dispersion}
\begin{equation}
H(f; L) = \frac{E(f; L)}{E(f, 0)} = e^{-1/2\beta_2(2\pi f)^2L}
\end{equation}
where $H(f; L)$ is fiber frequency response due to dispersion after $L$ meters, and $\beta_2 = -\frac{D(\lambda)\lambda^2}{2\pi c}$. 

Fiber attenuation can be included with the factor $e^{-\frac{1}{2}\frac{att(\lambda)L}{10^4}}$, where $att(\lambda)$ is the fiber attenuation at wavelength $\lambda$ in dB/km.

For SMF28 the fiber dispersion is specified in terms of the zero-dispersion ($\lambda_0$) wavelength and the dispersion slope ($S_0$):
\begin{equation}
D(\lambda) = \frac{S_0}{4}\bigg(\lambda - \frac{\lambda_0^4}{\lambda^3}\bigg), 1200~\text{nm} < \lambda < 1600~\text{nm}
\end{equation}

\subsection{Polarization-Mode Dispersion}
Following \cite{Ip2008}

The output $\bm{y}(t)$ after fiber propagation is given by
\begin{equation}
\bm{r}(t) = \bm{h}(t)\ast \bm{x}(t) + \bm{n}(t)
\end{equation}
where, $\bm{x}(t) = [x_1(t), x_2(t)]^T$, $\bm{n}(t) = [n_1(t), n_2(t)]^T$ and $\bm{h}(t)$ is the matrix representation of the fiber impulse response
\begin{equation}
\bm{h}(t) = \begin{bmatrix}
h_{11}(t) & h_{12}(t) \\
h_{21}(t) & h_{22}(t) \\
\end{bmatrix}
\end{equation}
In the frequency domain
\begin{equation}
\bm{H}(\omega) = \begin{bmatrix}
\cos\theta & -\sin\theta \\
\sin\theta & \cos\theta
\end{bmatrix}\begin{bmatrix}
e^{j\omega\tau_{DGD}/2} & 0 \\
0 & e^{-j\omega\tau_{DGD}/2}
\end{bmatrix}\begin{bmatrix}
\cos\theta & \sin\theta \\
-\sin\theta & \cos\theta
\end{bmatrix}
\end{equation}

The first matrix from the right rotates the input polarization state to the principal states of polarization (PSPs). The second matrix adds a differential group dealy (DGD) $\tau_{DGD}$ between the two PSPs. Finally, the last matrix from the right rotates the PSPs to the output state of polarization.

$\tau_{DGD}$ has a Maxwellian distribution whose mean $\bar{\tau}_{DGD}$ grows with the square of the fiber length. In SMF, typically, $\bar{\tau}_{DGD} = 0.1 \mathrm{ps/\sqrt{km}}$ \cite{Ip2008}. This value assumes the strongly coupled regime, which should be accurate for fibers larger than tens of meters.

This formulation can model both CD and PMD of any order. In simulation, we do CD and PMD separately, since for modeling PMD the fiber has to be broken into small sections.

Hence the fiber propagation is modeled in the frequency domain as
\begin{equation}
\bm{Y}(\omega) = \Big(\bm{H_N}(\omega)\ldots \bm{H_1}(\omega)\Big)\times e^{-\frac{1}{2}\beta_2\omega^2L}
\end{equation}
where each of the $N$ sections of the fiber has frequency response matrix $\bm{H_i}(\omega)$ with a random phase $\theta_i \in [-\pi, \pi]$.
\section{Receiver}
\subsection{Homodyne downconversion}
The received signal electric field $E_s(t)$ and the local oscillator electrical field $E_{LO}(t)$ have the following passband representation
\begin{align}
E_s(t) &= \mathrm{Re}\Big\{\sqrt{2}|A_s(t)|\exp\Big(j(\omega_st + \phi_s(t) + \phi_{pn}(t))\Big)\Big\} \\
E_{LO}(t) &= \mathrm{Re}\Big\{\sqrt{2}|A_{LO}(t)|\exp\Big(j(\omega_{LO}t + \phi_{LO}(t))\Big)\Big\}
\end{align}
where $|A_s(t)|$ and $|A_{LO}(t)|$ is the slow envelope of the signal and local oscillator electric field, respectively. The absolute value is just to indicate that this only accounts for the amplitude of the transmitted symbols. $\omega_s$ and $\omega_{LO}$ the carrier frequency of the signal and local oscillator electric field. $\phi_s(t)$ is the phase information. $\phi_{pn}(t)$ and $\phi_{LO}(t)$ are the phase noise components.

Hence, it follows that the baseband representation is 
\begin{align}
E_s(t) &= |A_s(t)|\exp\Big(j(\phi_s(t) + \phi_{pn}(t))\Big) \\
E_{LO}(t) &= |A_{LO}(t)|\exp\Big(j(\omega_{off}t + \phi_{LO}(t))\Big)
\end{align}
where $\omega_{off} =\omega_{LO}-\omega_s$

Considering that the 3-dB coupler has the following transfer matrix
\begin{equation}
\begin{bmatrix}
E_{out, 1} \\
E_{out, 2}
\end{bmatrix} = \frac{1}{\sqrt{2}}\begin{bmatrix}
1 & j \\
j & 1 \\
\end{bmatrix}\begin{bmatrix}
E_{in, 1} \\
E_{in, 2}
\end{bmatrix}
\end{equation}
the $90^o$ hybrid has the following transfer matrix
\begin{equation}
\begin{bmatrix}
E_1\\
E_2\\
E_3\\
E_4
\end{bmatrix} = \frac{1}{2}\begin{bmatrix}
1 & 1 \\
j & -j \\
j & -1\\
-1 & j
\end{bmatrix}\begin{bmatrix}
E{s} \\
E_{LO}
\end{bmatrix}
\end{equation}
where $E_1, E_2, E_3, E_4$ are the input electric fields to the photodiodes.

After balanced detection we have
\begin{align}
I_i(t) = |A_s(t)||A_{LO}(t)|\cos(w_{off}t + \phi_s(t) + \phi_n(t)) \\
I_q(t) = |A_s(t)||A_{LO}(t)|\sin(w_{off}t + \phi_s(t) + \phi_n(t))
\end{align}

Therefore, the signal power per real dimension for a quaternary signal is
\begin{align} \nonumber
P_{sig} &= \langle|I_i(t)|^2\rangle = \langle|I_q(t)|^2\rangle \\
&= \frac{P_{rec}P_{LO}}{2}
\end{align}
since $P_{rec} = \langle|A_s(t)|^2\rangle$, and $P_{LO} = \langle|A_{LO}(t)|^2\rangle$

If \textbf{two polarizations} are used
\begin{align} 
P_{sig} &= \frac{P_{rec}P_{LO}}{8}
\end{align}

One-sided PSD per real dimension of thermal and shot noise:
\begin{align}
S_{th} = N_0
S_{sh} = 2(2qRP_{PD})
\end{align}
where $N_0$ is the PSD of the trans-impedance amplifier and $P_{PD}$ is the incident power in each photodiode:
\begin{equation}
P_{PD} = |E_1|^2 = |E_2|^2 = |E_3|^2 = |E_4|^2 \approx \frac{P_{LO}}{8}
\end{equation}
for two polarizations.

Hence, the SNR of an ideal system with no intersymbol interfence and employing matched filtering is 
\begin{equation}
\mathrm{SNR} = \frac{2P_{sig}}{(S_{sh} + S_{th})R_s/2}
\end{equation}
$R_s$ is the symbol rate and $R_s/2$ is the one-sided noise bandwidth. The factor of 2 in the signal power appears because we have both in phase and quadrature signal components.

\begin{equation}
\mathrm{SNR} = \frac{2E_s}{N_0}, \qquad \frac{E_s}{N_0} = \log_2M\frac{E_b}{N_0}
\end{equation}

\section{Digital signal processing based receiver}

\subsection{Fractionally spaced linear equalizer}

Long-haul systems typically have a static equalization stage to compensate for CD and a dynamic equalization stage, which is adapted more often since PMD changes over time.

For this application bandwidth limitations due to modulator and other devices, chromatic dispersion, polarization-mode dispersion are assumed to be small enough so they can all be compensated by a single equalization stage. 

Assuming an oversampling ratio $r_{os} = p/q$, $p, q \in \mathrm{Z^*_+}$, which need not be integer, we can compensate for intersymbol interference (ISI) and demultiplex the two polarizations.

Samples at the equalizer input $\bm{X}[k]$, where $x_1[k], x_2[k]$ are the complex samples from the two polarizations. $k$ is at a rate $T_s/r_{os}$.

The equalizer output can be written as
\begin{align}
\hat{x}_{1}[k] = w^{(i)}_{11}[k]\ast x_1[k] + w^{(i)}_{12}[k]\ast x_2[k] \\
\hat{x}_{2}[k] = w^{(i)}_{21}[k]\ast x_1[k] + w^{(i)}_{22}[k]\ast x_2[k]
\end{align}

where $i = 1, \ldots, q$. We need a total of $q$ filters because in the case of non-interger oversampling ratio the samples do not align. For instance, If $r_{os} = 3/2$, we need two sets of filters, one for the odd samples and another for the even samples.

This can be simplified if PMD is small so that 

\begin{align}
\hat{x}_{1}[k] = w^{(i)}_{1}[k]\ast x_1[k] + w_{21}^{(i)}w^{(i)}_{2}[k]\ast x_2[k] \\
\hat{x}_{2}[k] = w^{(i)}_{2}[k]\ast x_2[k] + w_{12}^{(i)}w^{(i)}_{1}[k]\ast x_1[k]
\end{align}
where $i = 1, \ldots, q$. Note that in this case we only have two filters $w^{(i)}_{1}$ and $w^{(i)}_{2}$ per symbol. The terms $w_{21}^{(i)}$ and $w_{12}^{(i)}$ are cross terms, which are also determined adaptively. Hence the number of operations is approximately halved.

The number of required is sufficiently small that time-domain equalization is less complex than frequency-domain equalization.

\subsection{Constant Modulus Algorithm}

Constant Modulus Algorithm (CMA) is a blind equalization algorithm that relies on the fact that PSK or 4-QAM constellations have a constant modulus. 

For being a blind algorithm, CMA cannot separate the symbols. If there's a fixed phase error CMA will converge to that phase offset. If there's a mismatch between the transmitter and local oscillator, the CMA output will rotate at a rate that is the difference between the two frequencies. For this reason, the constellation diagram after CMA is typically a circle. 

\subsection{Frequency Recovery}
\cite{Hoffmann2008} 

Input signal is in the form

\begin{equation}
x[k] = x_ke^{j2\pi\Delta fk + \phi[k]}
\end{equation}
where $\Delta f$ is the frequency offset to be estimated.

\begin{equation}
\Delta f[k] = \Delta f[k](1-\mu) + \mu\frac{\mathrm{arg}\{(x[k]x^*[k-1]^4)\}}{8\pi T_s}
\end{equation}
where $\mu$ is adaptation constant and $T_s$ is the symbol time. The $\mathrm{arg}\{\cdot\}$ function may require unwrapping to prevent cycle slips. Biased unwrapping, which takes into account estimated frequency was shown to have better performance \cite{Hoffmann2008}.

\subsection{Phase Locked Loop}

At instant $k$ the symbol $Y_k$ has phase components $\phi_s(k) + \phi_p(k) + \phi_n(k)$, where $\phi_s(k)$ is the signal component and  $\phi_p(k), \phi_n(k)$ are the noise components due to phase noise and additive Gaussian noise, respectively. $\phi_p(k)$ is a Weiner process whose random steps have variance $2\pi(2\nu)T_s$, where $\nu$ is the laser linewidth of both transmitter laser and local oscillator. $T_s$ is the symbol period. $\phi_n(k)$ has a von Mises also known as circular normal distribution or Tikhonov distribution.

\begin{equation} \label{eq:Tikhonov-pdf}
	\phi_n(k) \sim p(x; \mu, \kappa) = \frac{1}{2\pi I_0(\kappa)}\exp(\kappa\cos(x-\mu))
\end{equation}
where $\mu, 1/\kappa$ are analogous to $\mu$ $\sigma^2$ in the normal distribution. $I_0(\kappa)$ si the modified Bessel function of order 0.

Assuming that the phase estimation process perfectly removes the signal component from the phase of $Y_k$, the input to the PLL filter is $\phi_p(k) + \phi_n(k)$.

Note: the distribution of the ratio of two zero-mean Gaussian random variable is a Cauchy random variable.

Given the open-loop frequency response of the digital-PLL (DPLL) loop $G(z) = \phi_{LO}(z)/\phi_s(z)$, we can express the phase error as
\begin{equation}
	\phi_e(z) = \frac{1}{G(z)+1}\phi_s(z) + \frac{G(z)}{G(z) + 1}\phi_n(z)
\end{equation}


\subsubsection{Continuous time}

The second-order PLL has a loop filter given by $F(s) = 2\xi\omega_n + \omega_n^2/s$, where $\xi$ is the damping factor, which is typically $\xi = \sqrt{2}/2$, and $\omega_n$ is the relaxation frequency, which must be optimized in order to reduce the phase error variance.

The open-loop response of the PLL is given by
\begin{equation}
G(s) = \frac{F(s)}{s}e^{-s\Delta_T}
\end{equation}
where $\Delta_T$ is the loop delay, which includes group delay of the filters as well as propagation delays. $G(s)$ is the transfer function between the local oscillator phase and the phase noise, which also includes AWGN component. 

When there's a frequency offset in addition to a phase error, the input to the PLL is a ramp. Hence, from the final value theorem we can determine the steady-state phase error
\begin{equation}
\phi_e(t\to\infty) = \lim_{s\to 0} \frac{sR(s)}{1 + G(s)} = s\frac{1}{s^2}\cdot\frac{s^2}{s^2 + 2\xi\omega_ns + \omega_n^2} = 0
\end{equation}
Therefore, in continuous-time, a PLL has zero error to a frequency offset.

\subsubsection{Discrete time}

We can obtain the open-loop transfer function by discretizing the continuous-time transfer function $G(s)$, either using a bilinear transformation or impulse invariance.

From bilinear transformation
\begin{equation}
G(z) = \frac{z^2(\xi\omega_nT_s + T_s^2\omega_n^2/4) + 2zT_s^2\omega_n^2/4 + T_s^2\omega_n^2/4 - \xi\omega_nT_s}{z^2 - 2z + 1} = \frac{P(z)}{(z-1)^2}
\end{equation}

Applying the final value theorem to a ramp input
\begin{align} \nonumber
\phi_e(n\to\infty) &= \lim_{z\to 1} \frac{(z-1)\frac{(z)}{(z-1)^2}}{1 + G(z)} = \lim_{z\to 1} \frac{z}{(z-1)(1 + G(z))} \\
&= \lim_{z\to 1} \frac{z(z-1)}{(z-1)^2 + P(z)} = 0
\end{align}
Hence, the DPLL implementation also achieves zero error for a ramp input i.e., frequency offset. However, since the delay in the phase estimation of an actual implementation of a DPLL is at least 1, the output constellation will be rotated by at least $f_{off}/R_s$ radians.

\subsection{Implementations issues}
Hardware implementation of DPLL at high rates rely heavily on parallelism and pipelining in order to reduce the complexity of the processing units. 

\begin{equation}
p = \frac{\Delta_{PE}}{T_s}
\end{equation}
where $p$ is the number of parallel stages and $\Delta_{PE}$ is the total time it takes to perform phase estimation.

\subsection{IIR filter implementation}
The closed-loop transfer function of the PLL loop filter is given by
\begin{align} \nonumber
H(z) &= \frac{\phi_{LO}(z)}{\phi_n(z)} = \frac{G(z)}{1 + G(z)} \\ \nonumber
&= \frac{z^2(\xi\omega_nT_s + T_s^2\omega_n^2/4) + 2zT_s^2\omega_n^2/4 + T_s^2\omega_n^2/4 - \xi\omega_nT_s}{z^2(\xi\omega_nT_s + T_s^2\omega_n^2/4 + 1) + 2z(T_s^2\omega_n^2/4-1) + T_s^2\omega_n^2/4 - \xi\omega_nT_s + 1} \\
& = \frac{b_0 + b_1z^{-1} + b_2z^{-2}}{1 + a_1z^{-1} + a_2z^{-2}}
\end{align}

\subsection{Phase Tracking}
\cite{Phase1995}

Given the signal $y[k]$ after CMA equalization. The phase tracking algorithm applies a phase shift $z(k) = y[k]e^{-j\Phi[k]}$, where
\begin{equation}
\Phi[k+1] = \Phi[k]-\mu\mathrm{Im}\{z[k]e^*[k]\}
\end{equation}
where $e^*[k] = z[k] - \hat{x}[k]$, where $\hat{x}[k]$ is a decision made based on $z[k]$.

\bibliographystyle{plain}
\bibliography{bib}

\end{document}